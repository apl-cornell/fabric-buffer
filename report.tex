\documentclass{article}

\usepackage{arxiv}

\usepackage[utf8]{inputenc} % allow utf-8 input
\usepackage[T1]{fontenc}    % use 8-bit T1 fonts
\usepackage{hyperref}       % hyperlinks
\usepackage{url}            % simple URL typesetting
\usepackage{booktabs}       % professional-quality tables
\usepackage{amsfonts}       % blackboard math symbols
\usepackage{nicefrac}       % compact symbols for 1/2, etc.
\usepackage{microtype}      % microtypography
\usepackage{lipsum}

\title{Smart Buffer Project Report}

\author{
  Samantha (Yuhan) Deng
  %% examples of more authors
   \And
 Kyrylo Chernyshov
  %% \AND
  %% Coauthor \\
  %% Affiliation \\
  %% Address \\
  %% \texttt{email} \\
  %% \And
  %% Coauthor \\
  %% Affiliation \\
  %% Address \\
  %% \texttt{email} \\
  %% \And
  %% Coauthor \\
  %% Affiliation \\
  %% Address \\
  %% \texttt{email} \\
}

\begin{document}
\maketitle

% \begin{abstract}
% \lipsum[1]
% \end{abstract}


% keywords can be removed
% \keywords{First \and Second keyword \and More}


\section{Introduction}
In the Two-Phase Commit protocol, after a worker sends its decision of
abort/commit to the store, the worker needs to wait for the store's reply. In
other words, the worker is waiting for a message whose content it already knows.
The waiting is necessary to acount for the asynchrony of rthe underlying
communication channel. However, assuming that the communication channel is
synchronous most of the time, the waiting is unnecessary most of the time.

If the worker does not block on the reply of the store, fter sending out the
commit message for a transaction, the worker can start an new transaction that
depends on the previous transaction immediately. From the worker's point of
view, this is a 1.5-Phase Commit. However, the messages may arrive the store in
an order different from the worker sends them, and the store needs to sort out
the order of message arrival. 

In order to enable the store to deal with out of order messages, we designed a
data structure called smart buffer on the store's side. In Section
\ref{sec:design}, we will present our detailed design of the data structure and
how the data structure is embedded in the framework. Section
\ref{sec:implementation} contains implementation of the data structure in a toy
example of worker-store setup and in Fabric. Finally, performance results are
shown in Section \ref{sec:performance}.

\section{Design}
\label{sec:design}

\subsection{Design Goal}
To illustrate the design goal, consider two scenarios.

\paragraph{Scenario 1} Given a worker $w$, a store $s$ and two transactions $t_1$
and $t_2$. Assume that $t_2$ depends on $t_1$, in other words, there exists some
object $o$ such that $t_1$ writes version $v$ of $o$ and $t_2$ reads version $v$
of $o$. If $w$ sends out commit message for $t_1$ and sends out prepare message
for $t_2$ without waiting for $s$'s reply for $t_1$, $s$ may receive the prepare
message of $t_2$ before receiving the commit message of $t_1$. Then, the store
needs to prepare for $t_2$ that reads $v$ of $o$, but it only has $v - 1$ of
$o$. 

One option is to make $s$ abort the prepare for $t_2$. However, if $s$ receives
the messages in an order the same as $w$ sends them, the prepare of $t_2$ won't
fail because of the unknown version $v$ of $o$. Thus, insteand of directly
aborting the prepare of $t_2$, $s$ should wait until it receives the commit
message of $t_1$ and updates $o$ to version $v$, and then start the prepare of
$t_2$. Thus, the data structure needs to allow the store block transaction
prepare until all unknown versions of objects of the transaction is learnt by
the store.

\paragraph{Scenario 2} Given a store $s$ and a transaction $t$. Assume that the
store needs a newer version of some object $o$ to start the prepare of $t$ and
$t$ also reads version $v$ of some other object $o'$. If $s$ already has version
$v'$ of $o'$, or $s$ commits version $v'$ of $o$ while waiting for the newer
version of $o$ and $v' > v$, the prepare of $t$ will always fail because of
version conflict. Then, the prepare of $t$ should be aborted immediately even if
the store $s$ has not seen the newer version of $o$. 

Thus, the data structure nees to allow the store aborts pending transaction
prepare as soon as version conflicts happen.

\subsection{Smart Buffer Design}
The Smart Buffer interface has three methods: $\mathtt{add}$, $\mathtt{remove}$
and $\mathtt{delete}$.
\begin{itemize}
  \item $\mathtt{add(tid, deps)}$: $\mathtt{tid}$ is the ID of a transaction.
  $\mathtt{deps}$ is a set of objects with corresponding versions that the
  transaction depends on (i.e. reads). The method adds the transaction with
  $\mathtt{tid}$ and depends on $\mathtt{deps}$ to the buffer.
  \item $\mathtt{remove(object)}$: $\mathtt{object}$ is some version of an
  object. The method let the buffer know this version of an object is known to
  the store.
  \item $\mathtt{delete(tid)}$: $\mathtt{tid}$ is hte ID of a transaction. The
  method deletes the transaction with $tid$ from the buffer.
\end{itemize}
For transactions in the buffer, if version conflicts happen before all objects
the transaction read are known to the store, the buffer will notify the store to
abort the transaction. If all objects the transaction read are known to the
store, the buffer will notify the store the store to prepare for the transaction
again.

\subsection{Smart Buffer's Usage in the Store}
When the store receives a transaction prepare message, it tries to prepare for
the transaction. If the perpare does not contain version conflict and has some
version of an object newer than the version the store has, the store adds the
transaction and the objects it read to the buffer. 

When the store receives a transaction commit message, for each object the commit
updates, the store remove the object from the buffer. 

When the buffer notify the store to perpare for some transaction, the store
starts the prepare for the transaction again and notifies the worker of the
result of the prepare. When the buffer notify the store to abort some
transaction, the store aborts the transaction and notifies the worker of the
result.

\section{Implementation}
\label{sec:implementation}
\subsection{NumLinkBuffer}

\section{Performance}
\label{sec:performance}

\end{document}
